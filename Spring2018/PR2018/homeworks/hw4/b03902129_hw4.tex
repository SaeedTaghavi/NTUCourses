% !TEX program = xelatex
\documentclass{article}
\usepackage{/Users/jay/LaTeX/cs}
\usepackage{xeCJK}

\newcommand{\hmwkClass}{Probability and Statistics, Spring 2018}
\newcommand{\hmwkTitle}{Homework 4}
\newcommand{\hmwkDueDate}{May 14, 2018}
\newcommand{\tb}{\textbf}

\begin{document}

\thispagestyle{empty}
\section*{\hmwkClass \\
    \normalsize{\hmwkTitle} \\
    \normalsize{DUE DATE: \hmwkDueDate}
}

\hfill{B03902129 \, 資工四 \, 陳鵬宇}

\begin{enumerate}
    \item [\textbf{4.2.2}]

    \begin{enumerate}[label=(\alph*)]
        \item 
            \begin{align*}
                \int_{-5}^7 2c(v + 5) dv & = 1 \\
                (\frac{v^2}{2} + 5v) \Big|_{-5}^7 & = 1 / 2c \\
                c & = 1 / 144.
            \end{align*}

        \item
            \begin{align*}
                \int_4^7 1 / 144 \cdot 2(v + 5) dv & = \frac{63}{144} = \frac{7}{16}.
            \end{align*}

        \item
            \begin{align*}
                \int_{-3}^0 1 / 144 \cdot 2(v + 5) dv & = \frac{7}{48}.
            \end{align*}
        \item
            \begin{align*}
                \int_{-5}^a 1 / 144 \cdot 2(v + 5) dv & = 1 / 3 \\
                a^2 + 10a - 23 & = 0 \\
                a & = -5 + 4\sqrt{3}.
            \end{align*}

    \end{enumerate}

    \item [\textbf{4.3.3}]

    $$
    \frac{d F_U(u)}{du} = f_U(u) =
    \begin{cases}
        0     & u < -5, \\
        1 / 8 & -5 \le u < -3, \\
        0     & -3 \le u < 3, \\
        3 / 8 & 3 \le u < 5, \\
        0     & u \ge 5.
    \end{cases}
    $$

    \item [\textbf{4.4.5}]

    \begin{enumerate}[label=(\alph*)]
    \item
        $$
        f_Y(y) =
        \begin{cases}
            1 / 2 & -1 \le y \le 1, \\
            0     & \text{otherwise}.
        \end{cases}
        $$

        $$\text E[Y] = \int_{-1}^1 y / 2 dy = 0.$$

    \item
        \begin{align*}
            \text E[Y^2] & = \int_{-1}^1 \frac{y^2}{2} dy = 1 / 3, \\
            \text{Var}[Y] & = \text E[Y^2] - \text E[Y]^2 = 1 / 3 - 0 = 1 / 3.
        \end{align*}
    \end{enumerate}

    \item [\textbf{4.5.10}]

    \begin{enumerate}[label=(\alph*)]
        \item 
        $$
        f_X(x) =
        \begin{cases}
            1 / 10 & -5 \le x \le 5, \\
            0 & \text{otherwise}.
        \end{cases}
        $$

        \item

        $$
        \int_{-5}^x \frac{1}{10} dy = \frac{x}{10} + \frac{1}{2}.
        $$

        $$
        F_X(x) =
        \begin{cases}
            0 & x < -5 \\
            \frac{x}{10} + \frac{1}{2} & -5 \le x \le 5 \\
            1 & x > 5.
        \end{cases}
        $$

        \item $$\text E[X] = 0.$$
        \item $$\text E[X^5] = \int_{-5}^5 \frac{1}{10} x^5 dx = 0.$$
        \item $$\text E[e^x] = \int_{-5}^5 e^x \frac{1}{10} dx = \frac{1}{10} (e^5 - e^{-5}).$$

    \end{enumerate}

    \item [\textbf{4.6.4}]

    \begin{enumerate}[label=(\alph*)]
        \item

        \begin{align*}
        \text P[Y \le 10] = 0.933 = \Phi(1.5) = \Phi(z) \\
        \Rightarrow z = 1.5 = \frac{x - \mu}{\sigma} = \frac{10 - \mu}{10} \Rightarrow \mu = -5.
        \end{align*}

        \item

        \begin{align*}
        \text P[Y \le 0] = 0.067 = 1 - 0.933 = 1 - \Phi(1.5) = 1 - \Phi(z) = \Phi(-z) \\
        \Rightarrow z = -1.5 = \frac{x - \mu}{\sigma} = \frac{0 - \mu}{10} \Rightarrow \mu = 15.
        \end{align*}

        \item

        \begin{align*}
        \text P[Y \le 10] = 0.977 \approx \Phi(1.99) = 0.9767 = \Phi(z) \\
        \Rightarrow z = 1.99 = \frac{x - \mu}{\sigma} = \frac{10 - \mu}{\sigma} \Rightarrow \mu = 10 - 1.99\sigma.
        \end{align*}

        \item

        \begin{align*}
        \text P[Y > 5] = 1 - F_Y(5) = \frac{1}{2} \Rightarrow \mu = 5.
        \end{align*}

    \end{enumerate}
    \item [\textbf{4.7.6}]

    $$
    F_X(x) =
    \begin{cases}
    1 - e^{-\lambda x} & x \ge 0, \\
    0 & \text{otherwise}.
    \end{cases}
    $$

    $$
    f_X(x) = \frac{dF_X(x)}{dx} =
    \begin{cases}
    \lambda e^{-\lambda x} & x \ge 0, \\
    0 & \text{otherwise}.
    \end{cases}
    $$

    $$
    \text E[X] = 3 = \frac{1}{\lambda} \Rightarrow \lambda = \frac{1}{3}.
    $$
    
    \begin{enumerate}[label=(\alph*)]
        \item

        Because $w = 60x$, $x = w / 60$.

        $$
        F_X(x) = 
        \begin{cases}
            1 - e^{-x / 3} & x \ge 0, \\
            0 & \text{otherwise}.
        \end{cases}
        $$

        $$
        F_W(x) = 
        \begin{cases}
            1 - e^{-w / 180} & w \ge 0, \\
            0 & \text{otherwise}.
        \end{cases}
        $$

        \item

        $$
        f_W(w) = \frac{dF_W(w)}{dw} =
        \begin{cases}
        0.2 + 0.3 = 0.5 & w = 0 \\
        \frac{1}{180} e^{-w / 180} & w > 0, \\
        0 & \text{otherwise}.
        \end{cases}
        $$

        \item

        \begin{align*}
            \text E[W] & = \int_0^\infty \frac{1}{180} e^{-w / 180} w dw & (\text{let } \frac{-w}{180}\Big|_0^\infty = t\Big|_0^{-\infty}) \\
                       & = \int_0^{-\infty} (-t)e^t (-180 dt) \\
                       & = 180.
        \end{align*}

        \begin{align*}
            \text{Var}[W] & = \text E[W^2] - (\text E[W])^2 \\
                          & = \int_0^\infty f(w) w^2 dw - (\text E[W])^2 \\ 
                          & = \int_0^\infty \frac{1}{180} e^{-w / 180} w^2 dw - 180^2 \\
                          & = 32400.
        \end{align*}
    \end{enumerate}
\end{enumerate}

\end{document}