% !TEX program = xelatex
\documentclass{article}
\usepackage{/Users/Jay/LaTeX/cs}
\usepackage{/Users/Jay/LaTeX/codelist}
\usepackage{/Users/Jay/LaTeX/xeCJK}

\newcommand{\hmwkClass}{Virtual Reality, Spring 2018}
\newcommand{\hmwkTitle}{Final Project Report}
\newcommand{\hmwkDueDate}{June 26, 2018}
\newcommand{\tb}{\textbf}

\begin{document}

\thispagestyle{empty}
\section*{\hmwkClass \\
    \normalsize{\hmwkTitle} \\
    \normalsize{DUE DATE: \hmwkDueDate}
}

\hfill{B03902129 \, 資工四 \, 陳鵬宇} \\

\section*{MonsterGo}

A VR game built with Unity.

\section*{Getting Started}

I use Unity to build this project.

To keep things as tiny as possible, I didn't include all files in this repo, but you can still build them step by step.

After cloning this repo, open the project with Unity. It'll import some necessary files.

Then navigate to Assets/DataFiles/Scenes, you can preview the game with different scenes here.

\section*{Dependencies}

The setup in my computer:

\begin{lstlisting}
Unity Hub                               Version 0.17.1
Unity                                   Version 2018.1.3f1 Personal
GVR SDK for Unity v1.130.1              GoogleVRForUnity_1.130.1.unitypackage
macOS High Sierra                       Version 10.13.5
Xcode                                   Version 9.3 (9E145)
\end{lstlisting}

\section*{Build Settings}

After cloning this repo, select File > Build Settings in the top bar.

Then choose the Platform you want to run and press Switch Platform. It'll take some time to import the assets. Here I use iOS as example.

Also, make sure that the order of \tb{Scenes In Build} in the Build Settings as follows:

\begin{lstlisting}
DataFiles/Scenes/MainScene              0
DataFiles/Scenes/DayScene               1
DataFiles/Scenes/NightScene             2
DataFiles/Scenes/EndScene               3
\end{lstlisting}

Then click the Player Settings.

In Settings for iOS -> Orientation -> Default Orientation*, select Landscape Left.

Otherwise, it'll cause unexpected behavior in your iPhone! (I haven't tested the behavior on Android Platform.)

\section*{Running the app}

In the top bar, select Build \& Run and press the Build And Run button in the bottom right, then save the project to any name you want.

Unity will then open the Xcode automatically, and make sure to choose the Team option (Here I choose Personal Team) in the Signing field.

Have fun with the MonsterGo!

\section*{Scripts Description}

There are three scenes in default:

\begin{itemize}
    \item MainScene: main menu
    \item DayScene: day mode
    \item NightScene: night mode
\end{itemize}




\end{document}