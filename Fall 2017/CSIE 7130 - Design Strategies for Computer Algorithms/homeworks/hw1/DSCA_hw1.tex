% !TEX program = xelatex
\documentclass[15pt]{extarticle}
\usepackage{/Users/jay/LaTeX/cs}
\usepackage{/Users/jay/LaTeX/codelist}

\newcommand{\hmwkClass}{演算法設計方法論(Design Strategies for Computer Algorithms)}
\newcommand{\hmwkTitle}{Homework 1}
\newcommand{\hmwkDueDate}{October 30, 2017}
\newcommand{\tb}{\textbf}

\begin{document}

\thispagestyle{empty}
\section*{\hmwkClass \\
    \normalsize{\hmwkTitle} \\
    \normalsize{DUE DATE: \hmwkDueDate}
}

% \begin{document} 
% \thispagestyle{empty}
% \fontsize{12pt}{12pt}\selectfont
% \section*{ \\
% \normalsize{} \\
% \normalsize{DUE DATE: }}

\hfill \textbf{學號:b03902129 \, 系級:資工四 \, 姓名:陳鵬宇} \\

\section{問題定義}

\textit{The string-to-string correction problem}是要解決以下的問題:\\

\begin{minipage}[t][][t]{0.06\textwidth}
\textbf{輸入}:
\end{minipage}
\begin{minipage}[t][][t]{0.9\textwidth}
字串$A=a_1a_2,\dots, a_{|A|}$和字串$B=b_1b_2,\dots,b_{|B|}$,
其中$|A|$和$|B|$ 分別為字串$A$和$B$的長度。\\
\end{minipage}

\begin{minipage}[t][][t]{0.06\textwidth}
\textbf{輸出}:
\end{minipage}
\begin{minipage}[t][][t]{0.9\textwidth}
將字串$A$ \textit{edit}至字串$B$ 花費最少的 sequense $S$, 其中 $S=s_1s_2,\dots,s_{|S|}$, $|S|$為$S$的長度,$S$有以三種\textit{edit}方式:
\begin{enumerate}
    \item 改變(\textit{change}): 將字串$A$的一個字元改變成另一個字元。
    \item 刪除(\textit{delete}):  將字串$A$的一個字元刪除。
    \item 插入(\textit{insert}): 在字串$A$中插入一個字元。
\end{enumerate}
\end{minipage}

\section{名詞、性質及定義}

%2.1
\subsection{Edit Distance}

%2.1.1
\subsubsection{{$A$}'s properties and notations}

先給定以下$A$的一些性質及表示法,方便讀者理解:
\begin{itemize}
    \item $A$為一個\uline{長度有限}的字串
    \item $a_i$為$A$的第$i$個字元
    \item $A_{i..j}=a_ia_{i+1},\dots, a_{j}$
    \item 當$i>j$時,$A_{i..j}$為空字串(null string)
    \item $|A|$為字串$A$的長度
    \item $A_0=A$
\end{itemize}

%2.1.2
\subsubsection{\textit{edit operation}}

\textbf{\uline{定義}}: 

\begin{center}
\textit{edit operation}: $a\to b$ ,若 $A=XaY$ 可透過 \textit{operation} $a \to b$ 變成 $B=XbY$ ($X$和$Y$為字串)。
\end{center}

我們用 $A\Rightarrow B$ \textit{via} $a\to b$ 表示之,其中$0\le|a|\le1$, $0\le|b|\le1$.

根據問題定義所敘述,同樣地,\textit{edit operation}: $a\to b$也有三種\textit{edit}方式:
\begin{enumerate}
    \item 改變(\textit{change}): change $a$ to $b$, 其中 $a\ne\varnothing$ 且$b\ne\varnothing$
    \item 刪除(\textit{delete}): delete $a$, 其中$b=\varnothing$ 
    \item 插入(\textit{insert}): insert $b$, 其中$a=\varnothing$
\end{enumerate}


%2.1.3
\subsubsection{\textit{edit sequence} $S$}

\textbf{\uline{定義}}:

\begin{center}
\textit{edit sequence} $S=s_1s_2,\dots,s_m$\mbox{ 且 }$AS = B$,
\end{center}

$s_i$是一個\textit{edit operation}: $a\to b$, $\forall i, 1\le i\le m$;
而$AS=B$代表:透過\textit{edit sequence} $S$, $A$會被\textit{edit}成$B$,如下所示:

\begin{align*}
    A &= A_0 \\
    A_0s_1 &= A_1 \\
    A_0s_1s_2=A_1s_2 &= A_2 \\
    &\vdots \\
    A_0s_1s_2,\dots,s_{m-1}=A_{m-2}s_{m-1}&= A_{m-1} \\
    A_0s_1s_2,\dots,s_{m-1}s_m=A_{m-1}s_m &= A_m=B \\
\end{align*}

此外,我們說 $S$ \textit{takes} $A$ to $B$,如果存在一個$S$ 可以將$A$ \textit{edit}成$B$

%2.1.4
\subsubsection{cost function $\gamma$ and \textit{edit distance} $\delta(A,B)$} 

\textbf{\uline{定義}}:

\begin{center}
cost function $\gamma(a\to b)$:對每一個\textit{edit operation} $a\to b$,我們賦與它一個\uline{非負}的實數。 
\end{center}

因為\textit{edit sequence} $S=s_1s_2,\dots,s_m$是由$m$個\textit{edit operations}所組成的,所以我們\textbf{\uline{定義}}:

\begin{center}
cost function $\gamma(S)=\sum_{i=1}^m\gamma(s_i)=\gamma(s_1)+\gamma(s_2)+\cdots+\gamma(s_m).$
\end{center}

但這樣的結果我們仍不滿意,我們希望能找到\uline{最小的} cost function,所以我們\textbf{\uline{定義}}:

\begin{center}
\textit{edit distance} $\delta(A,B)=\min\{\gamma(S)\mid$ \text{$S$ 為可以將$A$ edit成$B$的某一\textit{edit sequence}}\}.
\end{center}

我們題目的輸出就是希望能夠使上述的\textit{edit distance} $\delta(A,B)$擁有最小cost的$S$。

%2.2
\subsection{Traces $T$}

%2.2.1
\subsubsection{cost function of $T$}

\textbf{\uline{定義}}:

\begin{center}
cost function $\gamma(T_{A\to B})$,
\end{center}

其中$T_{A\to B}$為由$A$連到$B$的軌跡(\textit{trace}),它有以下兩種性質:
\begin{enumerate}
    \setlength\itemindent{1cm}
    \item [\textbf{(性質1)}] 對所有$T_{A\to B}$,存在一個\textit{edit sequence} $S$ \textit{taking} $A$ to $B$,使得:
    $$\gamma(S)=\gamma(T_{A\to B}).$$
    
    \item [\textbf{(性質2)}] 對所有\textit{edit sequence} $S$ \textit{taking} $A$ to $B$,存在一個$T_{A\to B}$,使得:
    $$\gamma(T_{A\to B})\le\gamma(S).$$
\end{enumerate}

所以我們知道 

\begin{center}
$\delta(A,B)=\min(\gamma(T_{A\to B}))$.
\end{center} 

\newpage
\section{解法敘述}
%3.1
\subsection{Computation of Edit Distance}
%3.1.1
\subsubsection{Dynamic Progamming}
% 給定$T_{A\to B}$, $A=A_1A_2$, $B=B_1B_2$, 並且假設$T_{A\to B}$中沒有任何一條線會由$A_i$連到$B_j$, $i\ne j$, $i,j\in\{1,2\}$,所以我們有:

% $$(T_{A\to B},A,B)=(T_1,A_1,B_1)\circ(T_2,A_2,B_2).$$

透過觀察,我們可以發現其實\textit{change}等於\textit{delete}後再\textit{insert},所以
\[\gamma(a\to b)=\gamma(a\to\varnothing)+\gamma(\varnothing\to b)\tag{*}
\]

現在要來尋找將$A$ \textit{edit}成$B$花費最少的$S$。
由式(*),我們給定以下的花費函數(cost function):

\begin{itemize}
    \item 改變(\textit{change}):$\gamma(a\to b)=\$2$, 其中$a\ne b$
    \item 刪除(\textit{delete}):$\gamma(a\to\varnothing)=\$1$
    \item 插入(\textit{insert}):$\gamma(\varnothing\to b)=\$1$
    \item 不變(\textit{doNothing}):$\gamma(a\to b)=\$0$, 其中$a=b$
\end{itemize}

將一個字串$A=a_1a_2\dots a_i$ \textit{edit}成字串$B=b_1b_2\dots b_j$的\textit{edit distance} $\delta(A,B)$用動態規劃的方式寫成:
\begin{align*}
    \delta(A,B)=\delta(A_{1..i},B_{1..j}) = \min\{& \delta(A_{1..i-1},B_{1..j-1})+\delta(a_i\to b_j), \\
    & \delta(A_{1..i-1},B_{1..j})+\delta(a_i\to\varnothing), \\
    & \delta(A_{1..i},B_{1..j-1})+\delta(\varnothing\to b_j)\}.
\end{align*}

舉例來說,假設我們有兩字串$A="alignment"$和$B=
algorithm"$,我們想要把$A$ \textit{edit}成$B$,

可以畫出如下的花費表格(\textit{doNothing}: $\nwarrow$, \textit{delete}: $\uparrow$, \textit{insert}: $\leftarrow$):

(因為\textit{change}是由\textit{delete}和\textit{insert}所組成的,所以我們省去此\textit{edit operation})

\begin{center}
\begin{tabular}{|c|c|c|c|c|c|c|c|c|c|c|c|} 
\hline
\backslashbox[1mm]{$A$}{$B$} & $\varnothing$ & a & l & g & o & r & i & t & h & m \\ 
\hline
$\varnothing$ & \cellcolor[gray]{0.8}$0,\emptyset$ & $1,\leftarrow$ & $2,\leftarrow$ & $3,\leftarrow$ & $4,\leftarrow$ & $5,\leftarrow$ & $6,\leftarrow$ & $7,\leftarrow$ & $8,\leftarrow$ & $9,\leftarrow$ \\
\hline
a & $1,\uparrow$ & \cellcolor[gray]{0.8}$0,\nwarrow$ & $1,\leftarrow$ & $2,\leftarrow$ & $3,\leftarrow$ & $4,\leftarrow$ & $5,\leftarrow$ & $6,\leftarrow$ & $7,\leftarrow$ & $8,\leftarrow$ \\
\hline
l & $2,\uparrow$ & $1,\uparrow$ & \cellcolor[gray]{0.8}$0,\nwarrow$ & $1,\leftarrow$ & $2,\leftarrow$ & $3,\leftarrow$ & $4,\leftarrow$ & $5,\leftarrow$ & $6,\leftarrow$ & $7,\leftarrow$ \\
\hline
i & $3,\uparrow$ & $2,\uparrow$ & \cellcolor[gray]{0.8}$1,\uparrow$ & $2,\leftarrow$ & $3,\leftarrow$ & $4,\leftarrow$ & $3,\nwarrow$ & $4,\leftarrow$ & $5,\leftarrow$ & $6,\leftarrow$ \\
\hline
g & $4,\uparrow$ & $3,\uparrow$ & $2,\uparrow$ & \cellcolor[gray]{0.8}$1,\nwarrow$ & $2,\leftarrow$ & $3,\leftarrow$ & $4,\leftarrow$ & $5,\leftarrow$ & $6,\leftarrow$ & $7,\leftarrow$ \\
\hline
n & $5,\uparrow$ & $4,\uparrow$ & $3,\uparrow$ & \cellcolor[gray]{0.8}$2,\uparrow$ & $3,\leftarrow$ & $4,\leftarrow$ & $5,\leftarrow$ & $6,\leftarrow$ & $7,\leftarrow$ & $8,\leftarrow$ \\
\hline
m & $6,\uparrow$ & $5,\uparrow$ & $4,\uparrow$ & \cellcolor[gray]{0.8}$3,\uparrow$ & $4,\leftarrow$ & $5,\leftarrow$ & $6,\leftarrow$ & $7,\leftarrow$ & $8,\leftarrow$ & $7,\nwarrow$ \\
\hline
e & $7,\uparrow$ & $6,\uparrow$ & $5,\uparrow$ & \cellcolor[gray]{0.8}$4,\uparrow$ & $5,\leftarrow$ & $6,\leftarrow$ & $7,\leftarrow$ & $8,\leftarrow$ & $9,\leftarrow$ & $8,\uparrow$ \\
\hline
n & $8,\uparrow$ & $7,\uparrow$ & $6,\uparrow$ & \cellcolor[gray]{0.8}$5,\uparrow$ & \cellcolor[gray]{0.8}$6,\leftarrow$ & \cellcolor[gray]{0.8}$7,\leftarrow$ & \cellcolor[gray]{0.8}$8,\leftarrow$ & $9,\leftarrow$ & $10,\leftarrow$ & $9,\uparrow$ \\
\hline
t & $9,\uparrow$ & $8,\uparrow$ & $7,\uparrow$ & $6,\uparrow$ & $7,\leftarrow$ & $8,\leftarrow$ & $9,\leftarrow$ & \cellcolor[gray]{0.8}$8,\nwarrow$ & \cellcolor[gray]{0.8}$9,\leftarrow$ & \cellcolor[gray]{0.8}$10,\leftarrow$ \\
\hline

\end{tabular}
\end{center}

由表格知$\delta(A,B)=\delta("alignment","algorithm")=10$;透過backtracking可寫出由$A$ \textit{edit}到$B$的過程:
\begin{align}
\text{alignment.$delete$(i)} &= \text{algnment}  \\
\text{algnment.$delete$(n)} &= \text{algment}  \\
\text{algment.$delete$(m)} &= \text{algent}  \\
\text{algent.$delete$(e)} &= \text{algnt}  \\
\text{algnt.$delete$(n)} &= \text{algt}  \\
\text{algt.$insert$(o)} &= \text{algot}  \\
\text{algot.$insert$(r)} &= \text{algort}  \\
\text{algort.$insert$(i)} &= \text{algorit}  \\
\text{algorit.$insert$(h)} &= \text{algorith}  \\
\text{algorith.$insert$(m)} &= \text{algorithm}  
\end{align}

當然這不是唯一解,在我的演算法中在計算$\delta$的$\min$要取哪一項若發生等於的情況,會優先選擇\textit{insert operation},但不論選擇哪一項符合$\min$的\textit{edit operation},$\delta(A,B)$都會等於10。\\

在這個例子中,$S=d_id_nd_md_ed_ni_oi_ri_ii_hi_m$, $|S|=10$, 其中$d$和$i$分別代表刪除(\textit{delete})和插入(\textit{insert}),下標的英文字母代表為對哪個字元作操作。\\

因為此演算法要將整個動態規劃的表格填滿,故時間複雜度為$O(|A|\times|B|)$。

%3.1.2
% \subsubsection{Application in BioEngineering}

% 我們可以將\textit{edit distance}的技術運用在生物基因\textit{string alignment}的問題上。

% \textit{string alignment} 的問題描述如下:\\

% \textbf{輸入}:$A=a_1a_2,\dots,a_m$和$B=b_1b_2,\dots,b_n$ \\

% \textbf{輸出}:$A=a_1a_2,\dots,a_m$和$B=b_1b_2,\dots,b_n$的最佳對齊方式下,得分$S_{m,n}$,評分標準如下($a$和$b$分別為$A$和$B$其中一個字元):
% \begin{enumerate}
%     \item Match: +8, $s(a,b)=8$ if $a=b$
%     \item Mismatch: -5, $s(a,b)=-5$ if $a\ne b$
%     \item Each gap symbol: -3, $s(a,-)=s(-,b)=-3$
% \end{enumerate}

% 對應到\textit{The string-to-string correction problem},
% \begin{enumerate}
%     \item Match:$s(a,b)=8\Rightarrow\gamma(a,b)=0$
%     \item Mismatch:$s(a,b)=-5\Rightarrow\gamma(a,b)=$ cost of \textit{change}
%     \item Each gap symbol:
%     \begin{itemize}
%         \item $s(a,-)=-3\Rightarrow\gamma(a,\varnothing)=$ cost of \textit{delete}
%         \item $s(-,b)=-3\Rightarrow\gamma(\varnothing,b)=$ cost of \textit{insert}
%     \end{itemize}
% \end{enumerate}

% 所以可得到一個類似\textit{edit distance}的動態規劃:
% \begin{align*} 
%     S_{i,j}=\max\{& S_{i-1,j-1}+S(a_i,b_j), \\
%     & S_{i-1,j}+S(a_i,-),\\
%     & S_{i,j-1}+S(-,b_j)\}.
% \end{align*}

\section{讀後心得}

在閱讀完$\langle$The String-to-String Correction Problem$\rangle$後,我對於字串之間的操作有了更進一步的了解,很佩服原作者Robert A. Wagner和Michael J. Fischer能夠在1974年這個資訊還不是很普及的時候,提出了這樣一篇強而有力的論文,他們提出\textit{edit distance}這個概念,讓字串間的操作有了新的理解,並且定義了花費函數(cost function),讓問題變得更加明確「找出$\delta(A,B)$和對應的$S$」。\\

在寫這份報告同時,也上網查了很多有關的例子,發現許多有趣的應用,像是生物的基因比對、git上面的diff指令、經典的\textit{Longest Common Subsequence(LCS)}問題等等。\\

此篇論文中有提到 $\delta(A,B)$ 的概念,說的白話一點就是我們不只希望能找到由$A$ \textit{edit}到$B$的過程($S$),同時我們還希望這個\textit{edit distance}是\textit{minimum}的,假設當我們看到以下三個基因序列 :

\begin{enumerate}
    \item AGCCT
    \item AACCT
    \item ATCT
\end{enumerate}

我們可能可以很直覺的說「2.和1.的相似程度比3.和1.還要高」,但能夠這樣用肉眼看出是因為我們舉的例子還很小,「肉眼」還感覺的出差異性,但一但我們的序列變得更長,不再是屬於同一個數量級別時,如何有一個有效的量化標準就變得很重要了,正好\textit{minimum edit distance}$\delta(A,B)$就能完成這件事。若要比較兩個基因序列$A="CTTAACT"
$和$B="CGGATCAT"$的相似程度,我們可以透過計算\textit{edit distance} $\delta(A,B)=\delta("CTTAACT","CGGATCAT")$,來做為一個標準,當$\delta(A,B)$較小時,這代表這兩個基因序列相似程度高;反之,代表相似程度低。\\

\textit{edit distance}演算法的時間複雜度是$O(|A||B|)=O(n^2)$,但人總是希望能夠做的更好,而2015年MIT的一篇論文$\langle$Edit Distance Cannot Be Computed in Strongly Subquadratic Time
(unless SETH is false)∗ $\rangle$就有提到這個時間複雜度是無法再被下降的。\\

在這個資訊蓬勃發展的二十一世紀,\textit{edit distance}或許已經稱不上什麼太高深的演算法,但是它歷久而彌新,在現今很多的應用中都能看見它的影子,並且默默的推動著Computer Science的發展。

\section{參考資料}
\begin{enumerate}[label={[\arabic*]}, noitemsep]
    \item \href{http://citeseerx.ist.psu.edu/viewdoc/download?doi=10.1.1.367.5281&rep=rep1&type=pdf}{$\langle$The String-to-String Correction Problem$\rangle$}
    \item \href{https://arxiv.org/pdf/1412.0348.pdf}{$\langle$Edit Distance Cannot Be Computed in Strongly Subquadratic Time
(unless SETH is false)∗$\rangle$}
    \item \href{https://en.wikipedia.org/wiki/Edit_distance}{Edit distance - Wikipedia}
    \item \href{https://en.wikipedia.org/wiki/Sequence_alignment}{Sequence alignment - Wikipedia} 
    \item \href{https://en.wikipedia.org/wiki/Levenshtein_distance}{Levenshtein distance - Wikipedia}
    \item \href{https://en.wikipedia.org/wiki/String-to-string_correction_problem}{String-to-string correction problem - Wikipedia}
    \item \href{https://blog.jcoglan.com/2017/02/12/the-myers-diff-algorithm-part-1/}{The Myers diff algorithm: part 1 – The If Works}
    \item \href{http://cpmarkchang.logdown.com/posts/222651-minimum-edit-distance}{自然語言處理 -- Minimum Edit Distance}
    
    
\end{enumerate}
\end{document}