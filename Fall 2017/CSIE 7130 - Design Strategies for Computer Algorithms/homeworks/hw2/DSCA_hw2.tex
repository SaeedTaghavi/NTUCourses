\documentclass[15pt]{extarticle}
\usepackage{cs}
    
\begin{document} 
\thispagestyle{empty}
\fontsize{12pt}{12pt}\selectfont
\section*{演算法設計方法論(Design Strategies for Computer Algorithms) \\
\normalsize{Homework 2} \\
\normalsize{DUE DATE: NOVEMBER 27, 2017}}

\hfill \textbf{學號:b03902129 \, 系級:資工四 \, 姓名:陳鵬宇} \\

% 1
\section{問題定義}
\textbf{問題}:\begin{minipage}[t]{0.8\linewidth}
    \textit{The weighted center of a tree.} \vskip0mm
\end{minipage}

\vskip3mm
\textbf{輸入}:
\begin{minipage}[t]{0.8\linewidth}
    一棵擁有$n$個點的樹$T=(V,E)$
    \begin{itemize}
        \item $d_{ij} = len(edge(i,j))\ge0,\forall$ edge$(i,j)$
        \item $w_i\ge0,\forall vertex$ $i$ 
        \item 我們可以連續的表示樹上的每一個點,舉例來說:點$x=(i,j;t)$代表$x$與點$i$的距離為$t$;與點$j$的距離為$d_{ij}-t$
    \end{itemize}
\end{minipage}

\vskip3mm
\textbf{輸出}:
\begin{minipage}[t]{0.8\linewidth}
\begin{itemize}
    \item 連續點解:重心$c$,使得$r(x)=\max(w_i d(x,i):i\in V)$ 為最小
    \item 離散點解:vertex $j$,使得$r(x)=\max(w_i d(x,i):i\in V)$ 為最小
\end{itemize}
\end{minipage}

% 2
\section{解法敘述}

% 2.1
\subsection{Top-down 觀點}
令$c$為$T$之重心,對所有鄰近$c$的$j$(以下用$adj(c)$表示之,為集合的概念),$|V_j(c)|\le n/2$($n$為$T$中點的數量),由[HM]我們知道$c$可以透過遍歷所有的點,在$O(n)$的時間內被找出來。\\

首先,我們對所有$j\in adj(c)$計算$r_j(c)$,這同時也會計算所有的$d(c,i)$[$i\in adj(c)$]。

\begin{itemize}
    \item 終止條件:當有兩點$j_1,j_2(j_1\ne j_2)$且$j_1,j_2\in adj(c)$,則:
        $r_{j_1}(c) = r_{j_2}(c)=r(c).$
        那麼重心$c$即為$center$,得解。
    \item 假設$j\in adj(c)$,且:
        $$r_j(c) > r_k(c), \forall k\in adj(c)\mbox{ 且 } k\ne j.$$
        那麼$center\in T_j(c)$。
\end{itemize}

% 2.2
\subsection{數學式推導}
當$u\notin V_j(c)$、$x\in T_j(c)$、$d(c, x) = t$時:
\begin{align*}
    d(u,x)  &= d(u,c) + d(c,x) = d(u,c) + t. \tag{1} 
            % &= d(u,c) + t. 
\end{align*}

如果$u,v\notin V_j(c)$,透過計算 
\begin{equation*}
    w_u(d(u,c)+t)=w_v(d(v,c)+t), \tag{2}
\end{equation*}
    
不失一般性我們可以假設:
\begin{equation*}
    w_ud(u,c)\ge w_vd(v,c). \tag{3}
\end{equation*}

$\exists t_{uv}(0\le t_{uv}\le\infty)$使得:
\begin{equation*}
    w_ud(u,x)\ge w_vd(v,x), \forall x\in T_j(c), d(x,c)=t \Leftrightarrow 0\le t\le t_{uv}. \tag{4}    
\end{equation*}

從以上的數學式我們可以知道:
\begin{enumerate}
    \item $d(center, c) < t_{uv}$:刪除點$v$(prune)
    \item $d(center, c) > t_{uv}$:刪除點$u$(prune)
\end{enumerate}


\newpage

% 2.3
\subsection{解法發想}
給定leaf $x$和$t$($t>0$, $t\in\mathcal R$),我們可以在$O(n)$的時間內找到點$y_1,\dots, y_l$使得$d(x,y_v)=t, v=1,\dots,l$.

\begin{enumerate}
    \item 計算所有的$d(x,i)$($i\in T$)
    \item $\forall$ edge$(i,j)$,如果$d(x,i)\le t\le d(x,j)$,那麼點$y_v\in$ edge$(i,j)$,$$d(y_v, i) = t-d(x,i)\Rightarrow d(x,y_v)=t$$ 
    \item $\forall z$, $d(x,z)\ge t$可以被表示成根為$y_1,\dots,y_l$的子樹(每一個點$y$可能貢獻$\ge2$子樹)
    \item 令這些子樹為$T_1,\dots,T_m$,根為$u_1,\dots,u_m(\{u_1,\dots,u_m\}\subseteq\{y_1,\dots,y_l\})$
    \item 令 $V_i$為$T_i$中所有點形成的集合,但不包含$\{u_i\}$,即$$V_i=V_i-\{u_i\}.$$
    \item 定義:$$R_i(x)=\max\{w_kd(x,k):k\in V_i\}.$$
    \item $\because V_i$兩兩不相交,$\therefore$ 我們可在$O(n)$時間內求出所有的$R_i(u_i), \forall i = 1,\dots,m$,分成以下三種情況討論:
    \begin{enumerate}
        \item $R_i(u_i) < R$: $c\notin T_i$。
        \item $R_{i_1}(u_{i_1})=R_{i_2}(u_{i_2})=R, i_1 \ne i_2$:$c\notin T_i,\forall i=1,\dots,m$。
        \item $R_i(u_i)=R$ 且 $i(1\le i\le m)$是唯一的:$c\in T_i$
    \end{enumerate}
    情況(c)等價於計算$r_j(y)$,其中$j\in adj(u_i)$ 且 $y=u_i$,這些步驟一樣可以在$O(n)$的時間算出。
\end{enumerate}

總括以上,可以在$O(n)$的時間內得知 
\begin{align*}
d(c,x) &\le t  \mbox{ 或}\\
d(c,x) &>   t 
\end{align*}

% 2.4
\subsection{解法流程}
原問題中,我們有樹$T$、重心$c$和子樹$T_j(c)$($c\in T_j(c)$)。

重新排列vertices $\notin T_j(c)$,即:$$(u_1,v_1),(u_2,v_2),\dots,(u_s,v_s)$$

$\because |v\in T_j(c)|\le n/2$,$\therefore s>\lfloor n/4\rfloor-1$($s$代表有$|s|$對pair),$\forall$ pair$(u,v)$,考慮2.2(2):
\begin{equation*}
    w_u(d(u,c)+t)=w_v(d(v,c)+t), \tag{2}
\end{equation*}

我們一樣可得2.2(3):
\begin{equation*}
    w_ud(u,c)\ge w_vd(v,c). \tag{3}
\end{equation*}

\begin{itemize}
    \item $w_u\ge w_v$:刪除點$v$(prune)。
    \item 否則:$t_{uv} = (w_u(d(u,c) - w_vd(v,c)) / (w_v - w_u).$
\end{itemize}

步驟如下:
\begin{enumerate}
    \item 計算所有的$t_{u_iv_i}$,直到不再有點會被prune。
    \item 尋找$t_{u_iv_i}, \forall i$的中位數,這可以在$O(n)$的時間內被求出,以實作上來看可以參考c++的std::nth\_element。
    \item 計算是否 $d(center, c) < t_m$($center\in T_j(c)$)
    \item 如果有一對pair$(u,v)$,使得$t_{uv}\ge t_m$,可以由:$$w_ud(u,x^*)\ge w_vd(v,x^*)$$ 來決定要刪除點$u$或是點$v$。
\end{enumerate}

\newpage
% 2.5
\subsection{時間複雜度}
由prune and search的過程,每一次prune會有大約一半的pair中的一個點被刪除,也就是約有$\frac{1}{2}\times\frac{1}{2}\times\frac{1}{2}n=\frac{1}{8}n$個點在樹$T$中被刪除掉,這產生了一個遞迴的新問題,考慮$T'$($T'$中點的數量大約是$T$的$7/8$倍),每次prune的過程我們假設花費$C(n)$的時間,可得以下遞迴式:$$T(n) \le T(\frac{7n}{8}) + Cn,$$

由 Master Theorem,我們知道 $T(n)=O(n)$。

% 2.6
\subsection{後記:離散版本}
點非連續的,即:$c'\in V$($c$:連續解的重心,$c'$:離散解的重心),由$r(x)$的凸性我們知道離散的解:
\begin{enumerate}
    \item $c\in V$:$c'=c$
    \item $c\in edge(i, j)$
    \begin{itemize}
        \item $c' = i$
        \item $c' = j$
    \end{itemize}
\end{enumerate}

\section{閱讀心得}
因為用到了prune and search的技巧,使得每一個步驟都能在線性的時間內被完成($O(n)$),是一個效率很高的演算法。\\

\textit{The weighted center of a tree}的應用層面很廣,例如當貨運公司,要決定自己的倉庫要放在一個國家中哪個位置,可以由每個可能被運送的城市(vertex)的消費強度($w_i,\forall vertex$ $i$)以及每個城市彼此間的交通費(edge$(i,j)$),由此計算出倉庫位置(重心$c$)。\\

當每個城市的消費強度相當時,題目會被簡化(reduce)成最小生成樹(\textit{minimum spanning tree}, MST),當然也可以在線性的時間內算出倉庫位置。

    

\end{document}