% !TEX program = xelatex
\documentclass{article}
\usepackage{/Users/jay/LaTeX/cs}
\usepackage{/Users/jay/LaTeX/xeCJK}

\newcommand{\hmwkClass}{Probability and Statistics, Spring 2018}
\newcommand{\hmwkTitle}{Homework 6}
\newcommand{\hmwkDueDate}{June 4, 2018}
\newcommand{\tb}{\textbf}

\begin{document}

\thispagestyle{empty}
\section*{\hmwkClass \\
    \normalsize{\hmwkTitle} \\
    \normalsize{DUE DATE: \hmwkDueDate}
}

\hfill{B03902129 \, 資工四 \, 陳鵬宇}

\begin{enumerate}
    \item [6.1.2]

    \begin{align*}
        P_W(-4) & = P_{X, Y}(-2, -1) = 3 / 14, \\
        P_W(-2) & = P_{X, Y}(-2, 0) + P_{X, Y}(0, -1) = 3 / 14, \\
        P_W(0)  & = P_{X, Y}(-2, 1) + P_{X, Y}(2, -1) = 2 / 14, \\
        P_W(2)  & = P_{X, Y}(0, 1) + P_{X, Y}(2, 0) = 3 / 14, \\
        P_W(4)  & = P_{X, Y}(2, 1) = 3 / 14.
    \end{align*}

    \item [6.3.6]

    Since $\text E[T] = \frac{1}{\lambda} = 200 \to \lambda = \frac{1}{200}.$

    We have $$f_X(x) = \begin{cases}
        \frac{1}{200} e^{-x / 200} & x \ge 0, \\
        0 & x < 0.
    \end{cases}$$

    \begin{enumerate}[label=(\alph*)]
        \item
        \begin{align*}
            \text P[C = 30]
            & = \int_0^{300} \frac{1}{200} e^{-x / 200} dx \\
            & = -(e^{-1.5} - 1) = 1 - e^{-1.5} \\
            & = 1 - 0.223130 \\
            & = 0.776869.
        \end{align*}

        \item

        \begin{align*}
            C & = 30 + 0.5(T - 300) \\
            T & = 2(C - 30) + 300 \\
              & = 2C + 240.
        \end{align*}

        We have
        
        $$f_C(c) = \frac{1}{200}e^{-\frac{1}{200}(2c + 240)}.$$

        Therefore,

        $$f_C(c) = \begin{cases}
            0.776869 & c = 30, \\
            \frac{1}{200}e^{-\frac{1}{200}(2c + 240)} & c > 30. \\
            0 & \text{otherwise}.
        \end{cases}
        $$

        \item

        \begin{align*}
        \text E[C]
            & = \int_{-\infty}^\infty c f_C(c) dc \\
            & = \int_{30}^\infty c \frac{1}{200}e^{-\frac{1}{200}(2c + 240)} dc \\
            & = \Big [100 \cdot 30e^{-(60 + 240) / 200} + 10000e^{-(60 + 240) / 200} \Big] \\
            & = 13000e^{-1.5} \\
            & = 13000(0.223130) \\
            & \approx 2900.69.
        \end{align*}      
    \end{enumerate}

    \item [6.3.8]

    \begin{enumerate}[label=(\alph*)]
        \item 
        \begin{align*}
            \text P[Y = 0.5] 
            & = \text P[0 \le X \le 1] \\
            & = \int_0^1 f_X(x) dx \\
            & = \int_0^1 x / 2 dx \\
            & = \frac{x^2}{4} \Bigg |_0^1 \\
            & = \frac{1}{4}.    
        \end{align*}

        \item 
        
        Split limit of $x$ in the form of $y$ in two parts as, $\frac{1}{2} < y \le 1$ and $1 < y \le 2$.

        Since $Y \ge \frac{1}{2}$, there fore it can conclude that for $y < \frac{1}{2}$,

        $$F_Y(y) = 0.$$

        And for $y \ge 2$,

        $$F_Y(y) = 1.$$
    
        CDF $F_Y(y)$ for limits $1 < y \le 2$ is

        \begin{align*}
            F_Y(y) & = P[X \le y] \\
                & = \int_0^y f_X(x) dx \\
                & = \int_0^y \frac{x}{2} dx \\
                & \frac{y^2}{4}.
        \end{align*}

        Hence the CDF of $Y$ can be written as,

        $$F_Y(y) = \begin{cases}
            0 & y < \frac{1}{2}, \\
            \frac{1}{4} & \frac{1}{2} \le y \le 1, \\
            \frac{y^2}{4} & 1 < y < 2, \\
            1 & y \ge 2.
        \end{cases}
        $$

    \end{enumerate}

    \item [6.3.10]
    \begin{enumerate}[label=(\alph*)]
        \item 
        For $y < 0$, $F_Y(y) = 0$. \\
        For $y > 36$, $F_Y(y) = 1$. \\
        For $0 \le y \le 36$, 
        
        \begin{align*}
        F_Y(y) & = \text P[Y \le y] = \text P[9X^2 \le y] \\
        & = \text P[X^2 \le \frac{y}{9}] \\
        & = \text P[-\frac{\sqrt y}{3} \le X \le \frac{\sqrt y}{3}] \\
        & = \int_{-\frac{\sqrt y}{3}}^{\frac{\sqrt y}{3}} f_X(x) dx \\
        & = \int_{-\frac{\sqrt y}{3}}^{\frac{\sqrt y}{3}} \frac{1}{4} dx \\
        & = \frac{\sqrt y}{6}.
        \end{align*}

        Thus, the CDF of $Y$ is

        $F_Y(y) =
        \begin{cases}
            0 & y < 0, \\
            \frac{\sqrt y}{6} & 0 \le y \le 36, \\
            1 & y > 36. 
        \end{cases}
        $

        The PDF of $Y$ is

        $f_Y(y) =
        \begin{cases}
            \frac{1}{12\sqrt y} & 0 \le y \le 36 \\
            0 & \text{otherwise}.
        \end{cases}
        $
        \item 

        $$F_W(w) = \begin{cases}
            \frac{1}{12 \sqrt w} & 0 \le y \le 16, \\
            16 & \text{otherwise}.
        \end{cases}
        $$
    \end{enumerate}

    \item [6.4.1]

    \begin{align*}
        F_V(v) 
            & = \text P[V \le v] \\
            & = \text P[\max(X, Y) \le v] \\
            & = \text P[X \le v, Y \le v] \\
            & = \int_0^v \int_0^v 6xy^2 dx dy \\
            & = v^5.
    \end{align*}

    $$F_V(v) = \begin{cases}
        0 & v < 0, \\
        v^5 & 0 \le v \le 1, \\
        1 & v > 1.
    \end{cases}
    $$

    PDF $f_V(v)$ is

    $$f_V(v) = \begin{cases}
        5v^4 & 0 \le v \le 1, \\
        0    & \text{otherwise}.
    \end{cases}
    $$

\end{enumerate}

\end{document}