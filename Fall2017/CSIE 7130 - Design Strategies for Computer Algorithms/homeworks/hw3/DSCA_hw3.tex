\documentclass[15pt]{extarticle}
\usepackage{cs}

\begin{document} 
\fontsize{12pt}{12pt}\selectfont
\section*{演算法設計方法論(Design Strategies for Computer Algorithms) \\
\normalsize{Homework 3} \\
\normalsize{DUE DATE: DECEMBER 15, 2017}}

\hfill \textbf{學號:b03902129 \, 系級:資工四 \, 姓名:陳鵬宇} \\

% 1
\section{問題定義}
\textbf{問題}:\begin{minipage}[t]{0.8\linewidth}
    \textit{A Personnel Assignment Problem Solved by the Branch-And-Bound Strategy} \vskip0mm
\end{minipage}

\vskip3mm
\textbf{輸入}:
\begin{minipage}[t]{0.8\linewidth}
    \begin{itemize}
        \item 一大小為$n$的線性有序集 Persons $P=\{P_1, P_2, \dots, P_n\}$
        \begin{itemize}
            \item $P_1 < P_2 < \dots < P_n$
            \item 我們可以想像這些人(persons)是由某些標準所衡量,例如身高、年齡、輩份等 
    \end{itemize}
        \item 一大小為$n$的集合 Jobs $J=\{J_1,J_2,\dots,J_n\}$
        \begin{itemize}
            \item 這些工作(jobs)是部分排序的
            \item 每個人必需被指派到一個工作,即$$\forall\mbox{person } P_i, 1\le i\le n,\exists f(P_i)=\mbox{some }J_i, 1\le i\le n.$$
            \item 並且要求:
            \begin{itemize}
                \item $f(P_i)\le f(P_j)\Rightarrow P_i\le P_j.$
                \item $i\ne j\Rightarrow f(P_i)\ne f(P_j).$
            \end{itemize}
        \end{itemize}
    \end{itemize}
\end{minipage}

\vskip3mm
\textbf{輸出}:
\begin{minipage}[t]{0.8\linewidth}
    最小的花費(Cost) = $\sum\limits_{i,j}C_{ij}X_{ij}$
    \begin{itemize}
        \item $C_{ij}$為將$P_i$指派$J_j$的花費
        \item 若$P_i$被指派到$J_j$:$X_{ij}=1$,否則$X_{ij}=0$
    \end{itemize}
\end{minipage}

% 2
\section{解法敘述}
% 2.1
\subsection{例子描述}
我們以下用一個例來說明之:
給定$n=4$,那麼輸入會有4位persons和4個已經被「部分」排序的jobs,以下為「部分」排序jobs的圖:

\begin{center}
    \begin{tikzpicture}[->,>=stealth',shorten >=1pt,auto,node distance=2cm,
        main node/.style={circle,draw}]
        \node[main node] (J1)               {$J_1$};
        \node[main node] (J2) [right of=J1] {$J_2$};
        \node[main node] (J3) [below of=J1] {$J_3$};
        \node[main node] (J4) [below of=J2] {$J_4$};

        \path
            (J1) edge   node {} (J3)
            (J1) edge   node {} (J4)
            (J2) edge   node {} (J4);            
    \end{tikzpicture}
\end{center}

透過觀察此有向圖,我們知道:
\begin{itemize}
    \item 因為$J_1$為$J_3,J_4$的父節點,所以$J_1$必在$J_3,J_4$前。 
    \item 因為$J_2$為$J_4$的父節點,所以$J_2$必在$J_4$前。    
\end{itemize}

\newpage
以下是所有可能的拓撲排序(topologically sorted sequences):
\begin{enumerate}
    \item $J_1,J_2,J_3,J_4$
    \item $J_1,J_2,J_4,J_3$
    \item $J_1,J_3,J_2,J_4$
    \item $J_2,J_1,J_3,J_4$
    \item $J_2,J_1,J_4,J_3$
\end{enumerate}

透過列出所有可能的拓撲排序,我們可以根據以下規則得到一樹狀圖表示之:
\begin{enumerate}
    \item 先選取其中任一個沒有父節點的節點,在我們上圖舉的例子為$J_1$或$J_2$。
    \item 該節點即為所有可能拓撲排序的第一個元素。
    \item 移除該節點,回到步驟1.遞迴地做直到所有節點都被移除。
\end{enumerate} 

\begin{center}
    \begin{forest}
        for tree = {l sep=1cm, 
                    s sep=1.5cm,
                    minimum width=0.8cm,
                    edge={->}, 
                    circle, draw, anchor=center, fit=rectangle}
        [0 [1 [2[3[4]][4[3]]] [3[2[4]]]] [2 [1[3[4]][4[3]]]]]
    \end{forest}
\end{center}

此外,我們會有如下的 $C_{ij}$表格: 
\begin{center}
    \begin{tabular}{|c|c c c c|}
        \hline
        \diagbox{Persons}{Jobs} & $J_1$ & $J_2$ & $J_3$ & $J_4$ \\ 
        \hline 
        $P_1$ & 29 & 19 & 17 & 12 \\ 
        $P_2$ & 32 & 30 & 26 & 28 \\ 
        $P_3$ &  3 & 21 &  7 &  9 \\ 
        $P_4$ & 18 & 13 & 10 & 15 \\ 
        \hline
    \end{tabular}
\end{center}

我們可以對每個列(行)同減該列(行)的最小值,使每一行至少有一個$0$,並且記錄所有被刪減的值的和(Total),這個動作很重要! 
\begin{enumerate}
    \item 先對每列分列同減該行最小值,即$12,26,3,10$
    \begin{center}
        \begin{tabular}{|c|c c c c|}
            \hline
            \diagbox{Persons}{Jobs} & $J_1$ & $J_2$ & $J_3$ & $J_4$ \\ 
            \hline 
            $P_1$ & 17 &  7 &  5 &  0 \\ 
            $P_2$ &  6 &  4 &  0 &  2 \\ 
            $P_3$ &  0 & 18 &  4 &  6 \\ 
            $P_4$ &  8 &  3 &  0 &  5 \\ 
            \hline
        \end{tabular}
    \end{center}
    \item 我們發現第2行任一元素皆不為$0$,再對該行同減最小值$3$,此時每一行皆至少有一個$0$
    \begin{center}
        \begin{tabular}{|c|c c c c|}
            \hline
            \diagbox{Persons}{Jobs} & $J_1$ & $J_2$ & $J_3$ & $J_4$ \\ 
            \hline 
            $P_1$ & 17 &  4 &  5 &  0 \\ 
            $P_2$ &  6 &  1 &  0 &  2 \\ 
            $P_3$ &  0 & 15 &  4 &  6 \\ 
            $P_4$ &  8 &  0 &  0 &  5 \\ 
            \hline
        \end{tabular}
    \end{center}
    Total = $12+26+3+10+3=54.$
    這就是此題的lower bound,因為每一個分支(即分別指派$P_1,P_2,P_3,P_4$)都會至少花費54單位。所以我們可以將這個值拿來當初始值(還未指派工作給任一人) 
\end{enumerate}

\tikzset{every label/.style={xshift=2ex, text width=6ex, align=right, 
inner sep=1pt, font=\footnotesize, text=blue}}

\begin{center}
    \begin{forest}
        for tree = {l sep=1cm, 
                    s sep=1.5cm,
                    minimum width=0.8cm,
                    edge={->},
                    circle, draw, anchor=center, fit=rectangle}
        [0,label=54 [1,label=71 [2,label=72[3,label=76[4,label=81]][4,label=78[3,label=78]]] [3,label=71[2,label=86[4,label=91]]]] [2,label=58 [1,label=64[3,label=68[4,label=73]][4,label=70[3,label=70,fill=lightgray]]]]]
    \end{forest}
\end{center}

透過檢查最右下分支的子節點3(花費$=70$)可以知道,任何比70單位花費更高節點的都不會是我們所求的解,故可以刪掉左邊節點1(花費$=71$)下面的所有子節點。
此時我們可以得到我們的upper bound $= 70$。
\begin{center}
    \begin{forest}
        for tree = {l sep=1cm, 
                    s sep=1.5cm,
                    minimum width=0.8cm,
                    edge={->}, 
                    circle, draw, anchor=center, fit=rectangle}
                    [0,label=54 [1,label=71,fill=lightgray] [2,label=58 [1,label=64[3,label=68[4,label=73]][4,label=70[3,label=70]]]]]
    \end{forest}
\end{center}

\newpage
如果我們在前面沒有先對所有的結點做減縮(reduce)的動作,考慮以下的指派工作:$$P_1\to J_1, C_{11}=29,$$ 

即始我們將所有工作都指派完畢得到了upper bound$=70$,也無法bound住$P_1\to J_1$,因為他的花費只有$29<70$,這也是為什麼一開始做減縮的動作這麼重要,可以幫助我們大副減少搜索的時間。

\section{閱讀心得}

這次的心得報告不像以往是閱讀paper,而是一本原文的演算法教科書。Branch and Bound和之前學習過的線性歸劃有些概念很相似,這些問題都是NP hard,且都在處理整數問題,但我們仍然希望能找到一個較好的解法,即便他不是多項式時間。\\

不論是Branch and Bound,或是之前曾經學到的Prune and Search,他們的目的都是希望能夠減少「不必要的」計算過程,那些可能是早早就已經被發現不影響結果的,例如比lower bound低,或是比upper bound高,這跟期中考第2題要去找出$x_{right},x_{left}$來刪去越界的線去加速整體計算時間有異曲同工之妙。\\

我覺得在學習演算法的過程中,是充滿愉快的,在Google相關資訊時,能夠挖掘到海量相關的內容,這種自我學習獲得知識的感覺真的很美好。每一個演算法就像是現實生活中某些事情的投影,我們致力於去降低時間複雜度、增加效率,現實生活中也是一樣,我們都希望事情能夠事半功倍,例如:當我們想念好演算法,就會用挑比較好的參考資料去學習之。像老師您喜歡品茗,也會用較好的茶具泡茶吧!

\end{document}